\documentclass{article}
\usepackage[margin=1in]{geometry}
\usepackage[tbtags]{amsmath} 				% split tags on lower right
\usepackage{amssymb, mathrsfs}			% for math script fonts
\usepackage{physics}
\usepackage{graphicx}			% for images
\usepackage{tensor}				% for tensors
\usepackage{slashed}				% for Dirac slash
\usepackage{bm}								% for bold Greek
\usepackage[colorlinks=true]{hyperref}	% for hyperlinks in contents

\usepackage[T1]{fontenc} 		% updated fonts (e.g. textcen has caps letters)

\setlength\parindent{0pt}

\newcommand{\stirling}[2]{\genfrac{[}{]}{0pt}{}{#1}{#2}}

\title{Project Euler Solutions}
\author{Kevin Chen}

\begin{document}

\maketitle

Solutions that are particularly enlightening are explained here. 
Questions labeled with a $\star$ can be solved by hand or can be evaluated by an explicit formula in terms of simple functions. \\

\subsection*{1$\star$ -- Multiples of 3 and 5} 
The sum of the first $n$ numbers is $n(n+1)/2$. 
So the sum of the multiples of $k$ under 1000 is $S_k = k n_k(n_k+1)/2$ where $n_k=\lfloor 999/k \rfloor$. 
The answer is $S = S_3 + S_5 - S_{15}$, or written explicitly,
\[ \boxed{ 3\frac{333 \cdot 334}{2} + 5\frac{199 \cdot 200}{2} - 15\frac{66 \cdot 67}{2} } \]

\subsection*{2$\star$ -- Even Fibonacci numbers} 
The closed-form expression for Fibonacci numbers is $F_n = (\varphi_+^n - \varphi_-^n)/\sqrt{5}$, where $\varphi_\pm = (1\pm\sqrt{5})/2$.
This can be proved by induction, using the fact that $\varphi_\pm$ are the solutions to $x^2 = x + 1$.
As $|\varphi_-| < 1$, for large $n$ we can approximate $F_n \approx \varphi_+^n/\sqrt{5}$. 
Using this, we can find indices $n$ where $F_n$ is less than four million. 
$4 \cdot 10^6 > F_n \approx \varphi_+^n/\sqrt{5}$ implies that $n < (\ln 4 + 6 \ln 10 + \ln \sqrt{5})/(\ln \varphi_+) \approx 33.263$. 
Therefore we are considering $1 \leq n \leq 33$. 
Every third Fibonacci number is even, so we are finding the sum $S = F_3 + F_6 + \dotso + F_{33}$. 
However, by definition of Fibonacci numbers, $S = (F_1 + F_2) + (F_4 + F_5) + \dotso + (F_{31} + F_{32})$ as well. 
Therefore $S = (F_1 + F_2 + F_3 + \dotso + F_{33})/2$. 
The sum of the first $n$ Fibonacci numbers is $F_{n+2} - 1$, so given that $F_{35} = 9227465$ the sum is $\boxed{ (9227465 - 1)/2 }$

\subsection*{5$\star$ -- Smallest multiple} 
This is easiest done by hand by writing down the prime factorizations for each number, and picking the smallest collection of prime divisors that will cover all numbers.

\subsection*{6$\star$ -- Sum square difference} 
The sum of the first $n$ numbers is $\frac{n(n+1)}{2}$. 
The sum of the squares of the first $n$ numbers is $\frac{n(n+1)(2n+1)}{6}$. 
Therefore the answer is,
\[ \boxed{ \left(\frac{100 \cdot 101}{2}\right)^2 -  \frac{100 \cdot 101 \cdot 201}{6} } \]

\subsection*{9$\star$ -- Special Pythagorean triplet} 
A \emph{primitive Pythagorean triplet} $a^2 + b^2 = c^2$ is one where $a$, $b$, and $c$ do not all share a common factor.
These are in a one-to-one correspondence with coprime integers $m$, $n$ with opposite parity and $0 < n < m$, where we take
\begin{align*}
a &= m^2 - n^2 & b &= 2mn & c &= m^2+n^2
\end{align*}
All Pythagorean triplets are integer multiples of primitive Pythagorean triplets, $(ka)^2 + (kb)^2 = (kc)^2$.
Therefore, for a generic triplet, if $a+b+c = 2km(m+n) = 1000$ then $km(m+n) = 500 = 5^3 2^2$. 
$m+n$ must be odd so we just check each case:

\begin{itemize}
\item[(i)] $m+n = 1$. We cannot satisfy $0 < n < m$. 
\item[(ii)] $m+n = 5$. We can have $m = 1, 2, 4$ but $n < m$ only allows $m = 4, n = 1, k = 25$. We have a solution $a = 200, b = 375, c = 425$.
\item[(iii)] $m+n = 25$. We can have $m = 1, 2, 4, 5, 10, 20$ but $n < m$ only allows $m = 20, n = 5$. However, this contradicts $m,n$ coprime.
\item[(iv)] $m+n=125$. We can have $m = 1, 2, 4$ but we cannot satisfy $n < m$. 
\end{itemize}

Therefore the answer is $\boxed{ 200 \cdot 375 \cdot 425 }$

\subsection*{10 -- Summation of primes} 
A very fast way of generating prime numbers up to some $N$ is to use a Sieve of Eratosthenes. 
Create a list of the first $N$ numbers, and start removing multiples of primes, starting at 2, then 3, then 5, and so forth.
We will be left with a table of primes.

\subsection*{12 -- Highly divisible triangular number} 
We need a fast way of counting divisors. 
Given an integer $n$ and its prime factorization, $n = p_1^{a_1} p_2^{a_2} \dotsb p_k^{a_k}$, the number of ways we can multiply the primes together gives the number of divisors of $n$. 
Thus the number of divisors of $n$ is $(a_1 + 1)(a_2 + 1)\dotsb(a_k+1)$. 

\subsection*{14 -- Longest Collatz sequence} 
A useful programming technique in dynamical programming is \emph{memoization}. 
In a process where you anticipate that a function would be called multiple times with the same input, it is often more efficient to remember a table of input-output values, which you should consult first.

\subsection*{15$\star$ -- Lattice paths} 
This is a combinatorics problem. 
The number of paths through the grid is equivalent to the number of ways of choosing 20 ``right'' moves from 40 total moves, setting the other moves to ``down''. 
Written explicitly, the answer is 
\[ \boxed{ \binom{40}{20} = \frac{40!}{20! \cdot 20!} } \]

\subsection*{17$\star$ -- Number letter counts} 
This problem is actually not toodious by hand. 
The numbers $1 \to 9$ use 36 letters and the numbers $10 \to 19$ use 70 letters. 
The numbers $20 \to 29$ then use $6\cdot10 + 36$ letters as ``twenty'' uses 6 letters. 
Continuing in this fashion, the numbers $1 \to 99$ use $(6+6+5+5+5+7+6+6)\cdot10 + 36\cdot9 + 70 = 854$ letters. 
Next, the numbers $100 \to 199$ use $(3+10)\cdot100 + 854 - 3$ letters. 
The $(3+10)$ factor comes from ``one hundred and'' using $3 + 10$ letters, and we subtract 3 letters at the end because we drop the ``and'' for 100. Continuing in this fashion, the numbers $1 \to 999$ use $(36 + 10\cdot9)\cdot100 + 854\cdot10 - 3\cdot9 = 21113$ letters. Then add 11 for ``one thousand'' to get a total of $\boxed{ 21124 }$.

\subsection*{19$\star$ -- Counting Sundays} 
Very interesting to use John Conway's Doomsday algorithm, but (un)surprisingly the answer is $1200/7$ rounded to the nearest integer.

\subsection*{23 -- Non-abundant sums} 
There is an interesting way to sum up the divisors of a number. 
Let $n$ be a number and suppose its prime factorization is $n = p_1^{a_1} p_2^{a_2} \dotso p_k^{a_k}$. 
Then $(1+p_1+p_1^2+\dotso+p_1^{a_1})(1+p_2+p_2^2+\dotso+p_2^{a_2}) \cdots (1+p_k+p_k^2+\dotso+p_k^{a_k})$ expanded would be the sum of all numbers that divide $n$. 
But $(1+p_i+p_i^2+\dotso+p_i^{a_i}) = \frac{p_i^{a_i+1}-1}{p_i-1}$, so 
\[ \sum_{d|n} d = \prod_{i=1}^k \frac{p_i^{a_i+1}-1}{p_i-1} \]

\subsection*{24$\star$ -- Lexicographic permutations} 
Our goal is to find the (unique) coefficients for $10^6 = a_9\cdot9! + a_8\cdot8! + \dotso + a_1\cdot1!$, which we can find using a long-division-like algorithm. 
$9!$ goes into $10^6$ two times, so $a_9$ is the $3^\text{rd}$ smallest unused digit, i.e. $a_9 = 2$. 
$10^6 - 2 \cdot 9! = 274240$, which $8!$ goes into 6 times.
So $a_8$ is the $7^\text{th}$ smallest unused digit, i.e. $a_8 = 7$. 
We repeat to create our permutation. 
However, because of indexing convention, we are looking for the coefficients of $10^6 - 1$ instead.

\subsection*{25$\star$ -- 1000-digit Fibonacci number} 
As explained in Problem 2, for large $n$ we can approximate $F_n \approx \varphi_+^n / \sqrt{5}$. 
So if some $F_n$ has a thousand digits, $10^{999} < F_n \approx \varphi_+^n / \sqrt{5}$ implies that $n > (999 \ln 10 + \ln \sqrt{5})/ \ln \varphi_+ \approx 4781.859$. 
This gives that $n = \boxed{4782}$ as the first Fibonacci number with more than a thousand digits.

\subsection*{26 -- Reciprocal cycles} 
If $n$ is not divisible by 2 or 5, then $1/n$ will have a repeating decimal representation with periodicity equal to the order of 10 modulo $n$, i.e.~the smallest integer $k \geq 1$ such that $10^k \equiv 1$ mod $n$. 
This can be checked using long-division. 

\subsection*{28$\star$ -- Number spiral diagonals} 
For a shell around the center with dimensions $n \times n$, the values at the corners are $n^2$, $n^2 - (n-1)$, $n^2 - 2(n-1)$, and $n^2 - 3(n-1)$. 
So we are summing $4n^2 - 6n + 6$ for odd $n$ from 3 to 1001. 
We can rewrite this slightly by letting $n = 2m+1$ so that now we are summing $16m^2 + 4m + 4$ for integer $m$ from 1 to 500. 
Now, recall that the sum of the first $m$ digits is $\frac{m(m+1)}{2}$ and the sum of the squares of the first $m$ digits is $\frac{m(m+1)(2m+1)}{6}$. 
Therefore the answer is 
\[ \boxed{ 16 \frac{500 \cdot 501 \cdot 1001}{6} + 4 \frac{500 \cdot 501}{2} + 4 \cdot 500 } \]

\subsection*{43$\star$ -- Sub-string divisibility} 
Rule 3 implies that $d_6 = 0$ or 5. 
But if $d_6=0$, then by Rule 5, $d_7=d_8$. 
Therefore $d_6 = 5$. 
Given that $500 \equiv 5$ mod 11, we have the following possibilities for $d_6d_7d_8$: 506, 517, 528, 539, 550 (ignored), 561, 572, 583, and 594. 
For each of these values, we check what value of $d_9$ allows $d_7d_8d_9$ to be divisible by 13. 
For example, take 506 and $060 \equiv 8$ mod 13, so $065$ is a possibility, but repeats 5 so it is ignored. 
Continuing this way, we get only four candidates for $d_6d_7d_8d_9$: 5286, 5390, 5728, and 5832. 
Repeating for $d_{10}$ for divisibility by 17, we get three candidates for $d_6d_7\dotsc d_{10}$: 52867, 53901, and 57289. \\

Next, we find $d_5$ such that $d_5d_6d_7$ is divisible by 7. 
We do this by recalling that $100 \equiv 2$ mod 7. 
Then for $d_6d_7 = 52$, $52 \equiv 3$ mod 7 so 252 and 952 are possible, but the first is ignored. 
For $d_6d_7 = 53$, $53 \equiv 4$ mod 7 so 553 is possible but is ignored. 
For $d_6d_7=57$, $57 \equiv 1$ mod 7 so 357 is possible. 
Thus we have two candidates for $d_5d_6\dotsc d_{10}$: 952867 and 357289. \\

Considering the first candidate, Rule 1 allows 0 and 4 as possible values for $d_4$. 
If $d_4 = 4$, then it is impossible to make $d_3d_4d_5$ divisible by 3 from our remaining digits 0, 1, and 3 (a number if divisible by 3 if its digits add up to 3). 
Therefore $d_4 = 0$, which forces $d_3=0$. 
$d_1$ and $d_2$ are 1 and 4 with no restrictions. \\

Considering the second candidate, possible values for $d_4$ are 0, 4, and 6. For the same reasons above, we can deduce that $d_3$ and $d_4$ are 0 and 6 with no restrictions, and likewise $d_1$ and $d_2$ are 1 and 4. \\

We have discovered six numbers that satisfy every condition, and their sum is
\[ \boxed{ (14+41)10^8 + 30952867 \cdot 2 + (1406+1460+4106+4160)10^6 + 357289 \cdot 4 } \]

\subsection*{45$\star$ -- Triangular, pentagonal, and hexagonal} 
Every hexagonal number is a triangle number, $T_{2n-1} = H_n$, so it suffices to solve $H_n = P_m$ or $n(2n-1) = m(3m-1)/2$ for positive integer solutions. 
After some rearrangement, we get $(6m-1)^2 - 3(4n-1)^2 = -2$. 
Letting $x = 6m-1$ and $y = 4n-1$, we are looking for positive integer solutions to the Pell equation $x^2 - 3y^2 = -2$.
Solutions are $(1+\sqrt{3})(2+\sqrt{3})^k$ for $k = 0, 1, 2, \dotsc$ (see Problem 94).
What remains is to check that $x_k \equiv 5$ mod 6 and $y_k \equiv 3$ mod 4. 
The answer can be reached after a few iterations.

\subsection*{63$\star$ -- Powerful digit counts} 
$10^n$ has $n+1$ digits, so we only need to consider bases 1 through 9. 
If we take 9, $9^n$ will never have more than $n$ digits, so the point where $9^n$ has less than $n$ digits is when $9^n < 10^{n-1}$. 
This simplifies to $n > \frac{1}{1 - \log_{10}9} \approx 21.854$. 
Thus for $1 \leq n \leq 21$, $9^n$ has $n$ digits. 
We can calculate this for each digit and add them up to get the answer. Written explicitly, the answer is 
\[ \boxed{ \sum\limits_{n=1}^9 \left\lfloor \frac{1}{1 - \log_{10} n} \right\rfloor } \]

\subsection*{64 -- Odd period square roots} 
The continued fraction expansion for an irrational square root $\sqrt{n}$ is a repeating non-terminating sequence $[a_0; \overline{a_1, a_2, \dotsc, a_p}]$.
The coeffcients $a_n$ can be calculated using the recursion relation,
\begin{align*}
m_0 &= 0 & d_0 &= 1 & a_0 &= \lfloor \sqrt{n} \rfloor \\
m_{n+1} &= d_na_n - m_n  &  d_{n+1} &= \frac{n - m_{n+1}^2}{d_n} & a_{n+1} &= \left\lfloor \frac{\sqrt{n} + m_{n+1}}{d_{n+1}} \right\rfloor 
\end{align*}
Then once $a_n = 2a_0$, one period has passed.

\subsection*{65 -- Convergents of e} 
Given a continued fraction expansion $[a_0; a_1, a_2, \dotsc]$, the convergents can be calculated by the recurrence relation,
\begin{align*}
p_{-1} &= 1 & q_{-1} &= 0 \\ 
p_0 &= a_0 & q_0 &= 1 \\ 
p_n &= a_n p_{n-1} + p_{n-2} & q_n &= a_n q_{n-1} + q_{n-2} 
\end{align*}
Then the fraction $p_n/q_n$ gives the $n$th convergent.

\subsection*{67 -- Maximum path sum II} 
Start from the bottom and work upwards.

\subsection*{69$\star$ -- Totient maximum} 
The totient of $n$ is given by $\phi(n) =n \prod (1-1/p_i)$, where the product is over all distinct prime factors $p_i$ of $n$. 
However, notice that $n/\phi(n) = \prod p_i/(p_i -1)$. 
Each term is greater than 1 and smaller primes give larger terms, so $n/\phi(n)$ is maximized by multiplying together small primes in increasing order. The largest product under 1 million is $\boxed{510510} = 2 \cdot 3\cdot5\cdot7\cdot11\cdot13\cdot17$.

\subsection*{72 -- Counting fractions} 
The number of reduced fractions with denominator $d$ is precisely the totatives of $d$. 
Therefore the answer is $$\sum\limits_{n=2}^{10^6} \phi(n)$$

\subsection*{76 -- Counting summations} 
This problem is basically asking for the integer partition of 100. 
This can be calculated using the recursive formula: $p(n) = \sum_k (-1)^{k-1} p(n - g_k)$ over all non-zero integers $k$, where $g_k = k(3k - 1)/2$ is the $k$th pentagonal number, $p(0) = 1$ and $p(x) = 0$ for $x < 0$.

\subsection*{79$\star$ -- Passcode derivation} 
Easier to do by hand. 

\subsection*{83 -- Path sum: four ways} 
We can solve this problem by an implementation of Dijkstra's algorithm. 
You traverse through the graph by keeping track of nodes you have visited and the minimum distance to that node, always advancing a step from the node with smallest distance.

\subsection*{84 -- Monopoly odds} 
We can model the monopoly board as a Markov chain. 
Then use a computer-algebra system to find eigenvectors or multiply the matrix over and over to get the transient behaviour.

\subsection*{86 -- Cuboid route} 
Here we present a new method for generating Pythagorean triples.
Given a primitive triple, we can generate three different primitive triples using the following linear transformations,
\[
\mqty[ -1 & 2 & 2 \\ -2 & 1 & 2 \\ -2 & 2 & 3 ] \mqty[ a \\ b \\ c ], \quad
\mqty[ 1 & 2 & 2 \\ 2 & 1 & 2 \\ 2 & 2 & 3 ] \mqty[ a \\ b \\ c ], \quad
\mqty[ 1 & -2 & 2 \\ 2 & -1 & 2 \\ 2 & -2 & 3 ] \mqty[ a \\ b \\ c ]
\]
Using an initial seed $(3, 4, 5)$ every primitive triple can be generated using this method.

\subsection*{94$\star$ -- Almost equilateral triangles} 
Given a triangle with side lengths $(a, a, a+i)$ where $i = \pm 1$, we can derive the area using Heron's formula to be $A = \frac{a+i}{4}\sqrt{(3a+i)(a-i)}$. 
In order to have an integer area, $(3a+i)(a-i) = y^2$ for some integer $y$.
We can rearrange this as $(3a-i)^2 - 3y^2 = 4$. 
If we let $x = 3a-i$, then we are solving the Pell equation $x^2 - 3y^2 = 4$. 
Solutions are $2(2+\sqrt{3})^{1+k}$ for $k = 0, 1, 2, \dotsc$.\footnote{For an explanation on how to sovle these equations, see \url{https://acollectionofelectrons.wordpress.com/2016/11/24/almost-equilateral-triangles-part-i/}}
What remains is if $x_k \equiv 1$ mod 3, then $i = -1$ and if $x_k \equiv 2$ mod 3, then $i = +1$. 
The answer can be reached after a few iterations.

\subsection*{100$\star$ -- Arranged probability} 
Let $b$ be the number of blue disks and $t$ be the total number of disks. 
We are looking for solutions where $\frac{b}{t} \cdot \frac{b-1}{t-1} = \frac{1}{2}$. 
Expanding and rearranging we can get $(2t-1)^2 - 2(2b-1)^2 = -1$. 
Letting $x = 2t-1$ and $y = 2b-1$ we get the negative Pell's equation $x^2 - 2y^2 = -1$. 
Solutions are $(1+\sqrt{2})^{1 + 2k}$ for $k = 0, 1, 2, \dotsc$ (see Problem 94).
Then $x, y \equiv 1$ mod 2 comes for free. 
The answer can be reached after a few iterations.

\subsection*{101 -- Optimum polynomial} 
The optimum polynomial $OP(k, n)$ is simply the polynomial $P$ interpolated on the points $n = 1, 2, \dotsc, k$ such that $P(n) = u_n$. 
This polynomial has degree $k-1$, so we are looking for the coefficients $P(n) = a_0 + a_1x^1 + \cdots + a_{k-1}x^{k-1}$. 
These coefficients can be found using linear algebra.
For instance, take $u_n = n^3$ and $k=3$, so $P(n) = a_0 + a_1x + a_2x^2$. 
Then we have the set of equations:
\[ \mqty{ a_0 + a_1\cdot1 + a_2\cdot1^2 = u_1 \\ a_0 + a_1\cdot2 + a_2\cdot2^2 = u_2 \\ a_0 + a_1\cdot3 + a_2\cdot3^2 = u_3 }
\quad\longleftrightarrow\quad \mqty[ 1 & 1 & 1 \\ 1 & 2 & 4 \\ 1 & 3 & 9 ] \mqty[ a_0 \\ a_1 \\ a_2 ] = \mqty[ 1 \\ 8 \\ 27 ] \]
We let $M$ be the matrix $M_{ij} = i^j$, $\mathbf{a} = (a_0, a_1, a_2)$ and $\mathbf{x} = (u_1, u_2, u_3)$. 
$M$ is called the \emph{Vandermonde matrix} and is invertible, so we have a unique solution $\mathbf{a}$ such that $M\mathbf{a} = \mathbf{x}$. 
This can be generalized for any $k$. \\

However, this method relies on the matrix inversion of large $k \times k$ matrices. 
A better method is \emph{Lagrange interpolation}. 
Given a set of points $\{x_1, x_2, \dotsc, x_k\}$ and values $\{u_1, u_2, \dotsc, u_k\}$, we find polynomials $P_i$ such that $P_i(x_j) = \delta_{ij}$. 
An explicit formula is:
$$ P_i(x) = \frac{\prod_{j \neq i}(x_{\phantom{i}} - x_j)}{\prod_{j \neq i}(x_i - x_j)}$$
Then our interpolation is simply $P(x) = \sum\limits_{i=1}^k u_i P_i(x)$.

\subsection*{106$\star$ -- Special subset sums: meta-testing} 
This problem can then be solved by hand, which we demonstrate by an example. 
It is useful to note that as a consequence of property (ii), we only need to test subset pairs of equal numbers of elements. 
Take $n=7$. 
First let us consider all disjoint subset pairs $A, B$ with 3 elements each. 
There are $\binom{7}{6} = 7$ ways to pick these 6 elements. 
Next, order our 6 elements in increasing magnitude, and for each element let ``$+$'' represent set-inclusion by $A$ and ``$-$'' represent set-inclusion by $B$. 
The corresponding string of ``$+$'' and ``$-$'' is defined to be our \textit{pattern}. 
For instance if our 6 elements are $\{1, 2, 3, 4, 5, 6\}$ and our pattern is $+-++--$, then $A = \{1, 3, 4\}$ and $B = \{2, 5, 6\}$. 
Without loss of generality we can assume that the smallest number belongs to $A$ and all patterns start with a ``$+$''. 
We then define the \textit{partial sums} of a pattern by taking $+ \mapsto 1$ and $- \mapsto -1$ and computing the partial sums left to right. 
For the same example above, the sequence of partial sums is $(1, 0, 1, 2, 1, 0)$. 
Subset pairs whose partial sums never become negative do not need to be tested for equality, as each element of $A$ is strictly less than a unique element from $B$ and $S(A) < S(B)$. 
Therefore, with 6 elements we only need to test subset pairs that have one of the 5 patterns: $++---+, +-+--+, +--++-, +--+-+, +---++$. 
Next, we consider subset pairs with 2 elements each, where there are $\binom{7}{4}=35$ different combinations and only 1 pattern: $+--+$. 
Therefore the total number of tests is $7\cdot5 + 35\cdot1 = 70$.\\

We can count the number of patterns to test for general $n$ and subset pairs with $k$ elements each (total $2k$ elements). 
The total number of patterns that start with a ``+'' is $\frac{1}{2}\binom{2k}{k}$. 
The number of patterns that have non-negative partial sums is given by the $k$-th Catalan number, $C_k = \binom{2k}{k} \frac{1}{k+1}$. 
Thus the number of paths to test is the difference, $\frac{1}{2}\binom{2k}{k} \frac{k-1}{k+1}$. 
This is multiplied by $\binom{n}{2k}$ and added for all $2 \leq k \leq n/2$. 
Written explicitly, the number of subset pairs to test is,
$$\boxed{ \frac{1}{2}\sum_{k=2}^{\lfloor{n/2}\rfloor} \binom{n}{2k} \binom{2k}{k} \frac{k-1}{k+1} }$$
In particular, for this question $n = 12$.

\subsection*{107 -- Minimal network} 
This can be solved using Kruskal's algorithm. 
You start with a graph with no edges and add edges one-by-one in increasing order of weight, skipping those that do not combine two trees, until your graph is connected.

\subsection*{108 -- Diophantine reciprocals I} 
$n < x, y$ so let $x = n+a$ and $y = n+b$. 
Then with some algebra we can show that $\frac{1}{n} = \frac{1}{n+a} + \frac{1}{n+b} \implies n^2 = ab$. 
Therefore the number of solutions is the number of ways to represent $n^2$ as the product of two numbers, or half its number of divisors.

\subsection*{113$\star$ -- Non-bouncy numbers} 
Given $d$ digits, there are $\binom{8+d}{8}$ increasing numbers. 
To derive this, you can work out that for $d$ nested sums and $k \geq 1$ that
$$\sum_{n_1=1}^k \left( \sum_{n_2=1}^{n_1} \left( \dotso \left(\sum_{n_d=1}^{n_{d-1}} 1\right) \dotso \right) \right) = \binom{k+d-1}{d} = \binom{k+d-1}{k-1}$$
and then letting $k=9$ for the digits 1 through 9. 
Alternatively, we can explain the formula inductively: for $d=1$, we have the set of digits $\{1, 2, 3, \dotsc, 9\}$, from which we choose one. 
For $d=2$, we have the set $\{1, 2, 3, \dotsc, 9, \#\}$ where $\#$ means that we double the first digit.
From this set, we pick two elements. 
For $d=3$, we have the set $\{1, 2, 3, \dotsc, 9, \#_1, \#_2\}$ where $\#_1$ means we double the first digit selected and $\#_2$ means we double the second digit selected, and we pick three elements.
This continues for higher $d$, so the number of increasing numbers is $\binom{8+d}{d}$.\\

The number of decreasing numbers is very similar as we are allowed digits 0 through 9, so for $d$ digits we have $\binom{9+d}{9}-1$ possibilities, minus 1 as we cannot start a number with 0. 
But we overcount numbers that are just repeating digits, which there are 9 of.
Thus for $d$ digits there are $\binom{8+d}{8} + \binom{9+d}{9} - 10$ non-bouncy numbers. 
If we sum this up for $1 \leq d \leq 100$, we get 
$$ \sum_{d=1}^{100} \binom{8+d}{8} + \sum_{d=1}^{100} \binom{9+d}{9} - \sum_{d=1}^{100} 10 = \left( \binom{8+100+1}{8+1} - 1 \right) + \left( \binom{9+100+1}{9+1} - 1 \right) - 1000 $$
$$ = \boxed{ \binom{109}{9} + \binom{110}{10} - 1002 }$$
Here we used the formula $\sum\limits_{m=0}^n \binom{m}{k} = \binom{n+1}{k+1}$.

\subsection*{114$\star$ -- Counting block combinations I} 
Let $f(n)$ be the number of ways to fill a row $n$ units in length. 
You can work out by hand that $f(n) = f(-1) + f(0) + f(1) + f(2) + \dotso + f(n-4) + f(n-1)$ where $f(-1) = f(0) = 1$. 
This is equivalent to defining $f(n) = 2f(n-1) - f(n-2) + f(n-4)$ where $f(0) = f(1) = f(2) = 1$ and $f(3) = 2$. 
The answer is $f(50)$, which can be calculated by hand.

\subsection*{115 -- Counting block combinations II} 
Nearly identical to Problem 114, but generalized. 
Now $F(m, n) = 2F(m, n-1) - F(m, n-2) + F(m, n-m-1)$ where $F(m, n) = 1$ when $n < m$ and $F(m, m) = 2$.

\subsection*{116$\star$ -- Red, green or blue tiles} 
The number of ways to tile with red tiles is $f_2(n)-1$, where $f_2(n) = f_2(n-1) + f_2(n-2)$ with initial values $f_2(1) = 1$ and $f_2(2) = 2$. 
These are precisely the Fibonacci numbers. 
Similarly, the number of ways to tile with green tiles is $f_3(n)-1$, where $f_3(n) = f_3(n-1) + f_3(n-3)$ with initial values $f_3(1) = f_3(2) = 1$ and $f_3(3) = 2$.  
Finally, the number of ways to tile with blue tiles is $f_4(n)-1$, where $f_4(n) = f_4(n-1) + f_4(n-4)$ with initial values $f_4(1) = f_4(2) = f_4(3) = 1$ and $f_4(4) = 2$. 
The answer is then $f_2(50) + f_3(50) + f_4(50) - 3$. 
It is possible to compute this value by hand.

\subsection*{117$\star$ -- Red, green, and blue tiles} 
The number of ways to tile is $f(n)$, where $f(n) = f(n-1) + f(n-2) + f(n-3) + f(n-4)$ with initial values $f(0) = 1$ and $f(n<0) = 0$. 
These are the tetranacci numbers. 
The answer is $f(50)$, which can be calculated by hand.

\subsection*{120$\star$ -- Square remainders} 
By expanding, you can determine that for odd $n$, $(a \pm 1)^n \equiv an$ mod $a^2$, and for even $n$, $(a \pm 1)^n \equiv 1$ mod $a^2$. 
Thus we are trying to maximize $2an$ mod $a^2$ for odd $n$, or equivalently finding largest even multiple of $a$ less than $a^2$. 
For odd $a$ this is $a^2 - a$, and for even $a$ this is $a^2 - 2a$. 
We are summing over all $3 \leq a \leq 1000$, so the answer is the sum of $(2k-1)^2 - (2k-1) + (2k)^2 - 2(2k) = 8k^2 - 10k + 2$ for $2 \leq k \leq 500$. 
Recall that the sum of the first $n$ numbers is $n(n+1)/2$ and the sum of the squares of the first $n$ numbers is $n(n+1)(2n+1)/6$. 
Therefore the answer is
$$\boxed{ \left(8\cdot\frac{500\cdot501\cdot1001}{6} - 10\cdot\frac{500\cdot501}{2} + 2\cdot500\right) - (8-10+2) }$$

\subsection*{121$\star$ -- Disc game prize fund}  
With $n=4$ turns, the number of ways to draw exactly $k=2$ red disks is 
\[ \sum_{1 \leq i < j \leq 4} i j = 1\cdot2 + 1\cdot3 + 1\cdot4 + 2\cdot3 + 2\cdot4 + 3\cdot4 \equiv \stirling{5}{3} \]
where $\stirling{a}{b}$ is the \emph{unsigned Stirling number of the first kind}.
In general, for $n$ turns, the number of ways to draw $k$ red disks is equal to $\stirling{n+1}{k+1}$.
Thus for $n = 15$ turns, the number of ways to win is $W = \stirling{16}{16} + \stirling{16}{15} + \cdots + \stirling{16}{9}$. 
There are $(n+1)!$ total outcomes, so if we award $x$ pounds for a win, our expected profit is $-(x-1)\cdot\frac{W}{16!} + 1\cdot\frac{16!-W}{16!}$. 
The break-even point is when our profit equals 0, which we can then solve for $x$ to get $x = \frac{16!-W}{W}+1 = \frac{16!}{W}$ and take the floor. 
Written explicitly, the answer is
$$\boxed{ \left\lfloor16! \left/ \sum_{k=9}^{16} \stirling{16}{k} \right. \right\rfloor}$$ 
The unsigned Stirling numbers of the first kind are easily generated using the recurrence relation $\stirling{n+1}{k} = n\stirling{n}{k} + \stirling{n}{k-1}$ with the initial conditions $\stirling{0}{0} = 1$ and $\stirling{0}{n} = \stirling{n}{0} = 0$ for $n > 0$. 
It may help to use the identity $n! = \sum_{k=0}^n \stirling{n}{k}$.

\subsection*{129 -- Repunit divisibility} 
We can generate $R(k)$ mod $n$ by the recurrence relation $R(k+1) = 10R(k) + 1$ with initial value $R(1) = 1$. 
There are finitely many numbers mod $n$, so $R(k)$ will eventually loop on itself. 
But because $n$ and 10 are coprime, the recurrence relation is invertible and creates a cycle. 
As $10\cdot0 + 1 = 1 = R(1)$, 0 is in our cycle and there exists a non-zero $k$ such that $R(k) \equiv 0$ mod $n$. 
Then $A(n)$ is the smallest such $k$. 
Additionally, as there are $n$ numbers and for one of then $R(k) \equiv 0$ mod $n$, we have $A(n) \leq n$. 
Therefore, for this problem we can start finding $A(n)$ from $n=10^6$. 

\subsection*{131 -- Prime cube partnership} 
We claim that $n$ is a perfect cube. 
Assume for a contradiction that $n = x \cdot y^3$ where $x \neq 1$ and is cubefree. 
As $n^3 + n^2p = n^2(n+p)$ is a perfect cube, so $x | n+p$. 
But $x|n$, so $x|p$ and $x=p$ as $p$ is prime. 
But then $n^2(n+p) = 2p^3y^6$ which is not a perfect cube. 
Therefore $n$ is a perfect cube, which implies that $n+p$ is a perfect cube as well. 
Therefore $p$ is the difference of two perfect cubes. 
But if $p = k^3 - (k-a)^3 = a(3k^2 - 3ka + a^2)$, we require $a=1$ and $p$ is the difference of two consecutive cubes. 
What remains is to generate consecutive cubes and test for primality.

\subsection*{132 -- Large repunit factors} 
$R(k) = (10^k-1)/9$, so if we are checking divisibility of $R(k)$ by some prime $p$, $(10^k-1)/9 \equiv 0$ mod $p$ if only if  $10^k \equiv 1$ mod $p$,  for $p \neq 3$. 
Therefore all we need to test is that $10^{10^9} \equiv 1$ mod $p$. 
We can simplify a little more using Fermat's little theorem, which says $10^{p-1} \equiv 1$ mod $p$. 
Therefore it suffices that we check $10^{\text{gcd}(10^9,\:p-1)} \equiv 1$ mod $p$.

\subsection*{137$\star$ -- Fibonacci golden nuggets} 
Here we work out the generating function for the Fibonacci numbers.
\begin{align*}
A_F(x) &= \sum\limits_{k=1}^\infty F_k x^k \\
&= F_1x + F_2x^2 + \sum\limits_{k=3}^\infty F_k x^k \\
&= x + x^2 + \sum\limits_{k=3}^\infty F_{k-1} x^k +  \sum\limits_{k=3}^\infty F_{k-2} x^k \\
&= x + x^2  - F_1x^2 + x\sum\limits_{k=1}^\infty F_k x^k +  x^2\sum\limits_{k=1}^\infty F_k x^k \\
&=  x + xA_F(x) + x^2A_F(x)
\end{align*}
So the generating function for the Fibonacci numbers if $A_F(x) = \frac{x}{1-x-x^2}$.
If we set $A_F(x) = n$, we can rearrange to get the equation $nx^2 + (n+1)x - n = 0$. 
By the quadratic formula, $x$ is rational when $(n+1)^2 + 4n^2$ equals a square number $y^2$. 
We can rearrange again to get $(5n+1)^2 + 4 = 5y^2$. 
Letting $x=5n+1$, we are solving the Pell equation $x^2 - 5y^2  = -4$. 
Solution are $2(1/2 + \sqrt{5}/2)^{1 + 2k}$ for $k = 0, 1, 2, \dotsc$ (see Problem 94). 
What remains is to find $x_k \equiv 1$ mod 5, which occurs when $k \equiv 0$ mod 2.
The answer can be reached after a few iterations.

\subsection*{138$\star$ -- Special isosceles triangles} 
Suppose we have a triangle with sides lengths $L, L, 2b$ and height $h = b + i$ perpendicular to side $2b$ where $i = \pm 1$. 
From the Pythagorean Theorem we have $b^2 + (2b+i)^2 = L^2$, which can be rearranged to $(5b+2i)^2+1 = 5L^2$. 
Letting $x = 5b+2i$ and $y= L$ we are solving the negative Pell equation $x^2 - 5y^2 = -1$. 
Solutions are $(1/2 + \sqrt{5}/2)^{3(1 + 2k)}$ for $k = 0, 1, 2, \dotsc$ (see Problem 94). 
$x_k \equiv 2$ or 3 mod 5 comes for free. 
The answer can be reached after a few iterations.

\subsection*{139$\star$ -- Pythagorean tiles} 
Suppose $a^2 + (a+k)^2 = c^2$ is a primitive Pythagorean triplet with $k > 0$. 
We are looking for cases where $k$ divides $c$. 
We have $2a^2 + 2ak + k^2 = c^2 \implies 2a^2 = c^2 - k(2a +k)$ but as $k|c$, we have $k=1$, $k=2$, or $k|a$. 
The second is impossible as $a$ and $a+k$ would have the same parity, but for primitive triplets exactly one of $a$ and $a+k$ is odd. 
The third is impossible as $k$ divides $a$, $a+k$, and $c$ so the triplet is not primitive. 
Therefore $k=1$ and we are searching for primitive triplets with legs that differ by 1. \\

Suppose $a^2 + (a+1)^2 = c^2$. 
We can rearrange this as $(2a+1)^2 + 1 = 2c^2$. 
Letting $x = 2a+1$ and $y = c$, we are solving the negative Pell equation $x^2 - 2y^2 = -1$. 
Solutions are $(1+\sqrt{2})^{1+2k}$ for $k = 0, 1, 2, \dotsc$ (see Problem 94). 
Then $x_k \equiv 1$ mod 2 comes for free. 
The answer can be reached after a few iterations. 
But here we only generate primitive triplets, so once we find the perimeter $P$ there are $\lfloor 10^8/P \rfloor$ triangles corresponding to a particular primitive triplet.

\subsection*{140$\star$ -- Modified Fibonacci golden nuggets} 
Using the same steps as Problem 137, the generating function is $A_G(x) = x(1+3x)/(1-x-x^2)$. 
We are eventually solving the Pell equation $x^2 - 5y^2 = 44$ for $x = 5n+7$. 
There are two fundamental solutions, $(8 + 2\sqrt{5})$ and $(7 + \sqrt{5})$, and further solutions are generated by multiplying powers of $(3/2 + \sqrt{5}/2)$ (see Problem 94). 
The answer can be reached after a few iterations.

\subsection*{145$\star$ -- How many reversible numbers are there below one-billion?} 
Let us define two types of pairs of digits with opposite parity: $\alpha-$pairs, where the two digits sum less than 10, and $\beta-$pairs, where the two digits sum greater than 10. 
There are 30 $\alpha-$pairs and 20 $\beta-$pairs, but only 20 $\alpha-$pairs if you exclude those that contain the digit 0. 
Next let $x$ be any number from 0 to 4, which there are 5 of. 
Then the table below summarizes the allowed patterns for reversible numbers. \\

$\quad \begin{array}{l | l | l} \text{\# of digits} & \text{Pattern} & \text{Count} \\ \hline
1 & - & 0 \\
2 & \alpha\alpha & 20 \\
3 & \beta x \beta & 20\cdot 5 \\
4 & \alpha\alpha\alpha\alpha & 20 \cdot 30 \\
5 & - & 0 \\
6 & \alpha\alpha\alpha\alpha\alpha\alpha & 20 \cdot 30^2 \\
7 & \beta x \beta x \beta x \beta & 20^2 \cdot 5^3 \\
8 & \alpha\alpha\alpha\alpha\alpha\alpha\alpha\alpha & 20 \cdot 30^3 \\
9 & - & 0 \end{array}$ \\

Therefore the answer is $\boxed{20 + 20\cdot 5 + 20 \cdot 30 + 20 \cdot 30^2 + 20^2 \cdot 5^3 + 20 \cdot 30^3}$.

\subsection*{148$\star$ -- Exploring Pascal's triangle}
It is best to start by drawing the non-multiples of 7 on Pascal's triangle.
Rows 1--7 form a triangle with 28 non-multiples of 7.
Call this $X_1$.
Rows 1--14 form a ``tri-force'' symbol, containing three copies of $X_1$.
This pattern continues for rows 1--49, where we have 28 copies of $X_1$ forming a ``generalized tri-force'' with seven triangles on its bottom layer.
Call this $X_2$, which contains a total of $28^2 = 784$ non-multiples of 7.
This pattern continues again, where rows 1--98 form a tri-force symbol, containing three copies of $X_2$, and so forth. \\

Using this picture, we can come up with an algorithm of computing the number of non-multiples of 7 in the first $10^9$ rows.
Let us list the first 7 triangle numbers:
\begin{align*}
T_0 &= 0 & T_1 &= 1 & T_2 &= 3 & T_3 &= 6 & T_4 &= 10 & T_5 &= 15 & T_6 &= 21
\end{align*}
First, we factorize $10^9$ in base 7: $10^9 = 33531600616_7$.
The highest digit is $3 \times 7^{10}$, so we expect there to be $T_3 = 6$ completed $X_{10}s$, and $3+1 = 4$ incomplete $X_{10}$s.
the next highest digit is $3 \times 7^9$, so each of these incomplete $X_{10}$s contains $T_3 = 6$ completed $X_9$s, and $3+1$ incomplete $X_8$s.
This continues, and the total number of non-multiples of 7 is
\[ \boxed{ 28^{10} T_3 + 4 \qty( 28^9 T_3 + 4 \qty( 28^8 T_5 + 6 \qty( 28^7 T_3 + 4 \qty( 28^6 T_1 + 2 \qty( 28^5 T_6 + 7 \qty( 28^2 T_6 + 7 \qty\big( 28 T_1 + 2 \qty\big( T_6 )))))))) } \]


\subsection*{162$\star$ -- Hexadecimal numbers} 
For a given number of digits $n$, we can count the number of possible hexadecimal numbers using the inclusion-exclusion principle. 
If boldface \textbf{1} denotes ``the set of numbers that contains at least one 1'', and likewise for \textbf{0} and \textbf{A}, then $|\mathbf{0} \cap \mathbf{1} \cap \mathbf{A}| = |\mathbf{0}| + |\mathbf{1}| + |\mathbf{A}| - |\mathbf{0} \cup \mathbf{1}| - |\mathbf{0} \cup \mathbf{A}| - |\mathbf{1} \cup \mathbf{A}| + |\mathbf{0} \cup \mathbf{1} \cup \mathbf{A}|$.
Add this up for $3 \leq n \leq 16$ to get the solution, 
\[ \boxed{ \sum_{n=3}^{16} \left(15 \cdot 16^{n-1} - 43 \cdot 15^{n-1} + 41 \cdot 14^{n-1} - 13^n\right) } \]

\subsection*{169 -- Exploring the number of different ways a number can be expressed as a sum of powers of 2} 
The function $f(n)$ is equivalent to the number of ways of expressing $n$ as a binary string, but also permitting the usage of the digit 2. 
For instance, $f(10) = f(1010_2) = \{1010, 210, 1002, 202, 122\}$ and $f(11) = f(1011_2) = \{1011, 211\}$. 
We can establish two properties of the function $f(n)$:
 \[ f(2n) = f(n) + f(n-1), \quad f(2n+1) = f(n) \]
\begin{itemize}
\item if $n$ is even (i.e. $n = 2k$) its binary string ends with a 0. Then its binary string representations are those of $k$ with a 0 digit appended at the end and those of $k-1$ with a 2 digit appended. For instance, $f(10) = f(1010_2) = \{1010, 210, 1002, 202, 122\} = \{101, 21\}_0 + \{100, 20, 12\}_2 = f(101_2) + f(100_2) = f(5) + f(4)$.
\item if $n$ is odd (i.e. $n = 2k+1$) its binary string ends with a 1. Then its binary string representations are those of $k$ with a 1 digit appended. For instance, $f(11) = f(1011_2) = \{1011, 211\} = \{101, 21\}_1 = f(5)$.
\end{itemize}
With this, we can recursively find $f(n)$ for all $n$.

\subsection*{183 -- Maximum product of parts} 
For a given $N$, we want to maximize $(N/k)^k$. 
We can do this by differentiating with respect to $k$ to get $(N/k)^k[\ln(N/k) - 1]$, so the maximum is when $k = N/e$. 
But this is not an integer, so we test the integers above and below for the larger value of $(N/k)^k$. 
Once we know our $k$, if $k/\text{gcd}(k, N)$ contains prime factors other than 2 or 5, then the decimal does not terminate.

\subsection*{188 -- The hyperexponentiation of a number} 
We are finding $1777\uparrow\uparrow1855 \equiv 1777^{1777\uparrow\uparrow1854}$ mod $10^8$. 
The generalized Fermat's little theorem says that $1777^{\phi(n)} \equiv 1$ mod $n$, where $\phi(n)$ is the totient function. 
So $1777^{1777\uparrow\uparrow1854} \equiv 1777^{k\cdot\phi(10^8) + a_1} \equiv 1777^{a_1}$ for some integer $a_1 \equiv 1777\uparrow\uparrow1854$ mod $\phi(10^8)$. 
We can repeat this process by taking successive totients, defining a sequence $a_1, a_2, \dotsc$. 
Eventually we will reach the totient $\phi^{(n)}(10^8) = 2$. 
As 1777 is odd, any power of 1777 is odd so $a_n = 1$. 
We can then work backwards using the recursive formula $a_{i} \equiv 1777^{a_{i+1}}$ mod $\phi^{(i)}(10^8)$ until we find $a_0$, which is the answer.

\subsection*{197 -- Investigating the behaviour of a recursively defined sequence} 
The function given is very close to the analytic function $f(x) = 1.42\cdot2^{-x^2}$. 
Here we are looking for stable fixed-points for the function $f(x)$, if they exist, i.e. points $x_0$ where $f(x_0) = x_0$ and $|f'(x_0)| < 1$. 
There is one fixed-point for $f(x)$ at $x_0 \approx 0.855$, but $f'(x_0) \approx -1.014$ so it is not stable. 
Next we have to look for fixed-points of the double map, $f\big(f(x)\big)$, which are $x_0 \approx 1.029$ and $x_1 \approx 0.681$. 
These fixed-points are stable and satisfy $f(x_0) = x_1$ and $f(x_1) = x_0$. 
If an initial point $u_0$ is picked near these values, iteratively calling $u_{n+1} = f(u_n)$ will converge upon these two fixed-points.

\subsection*{216 -- Investigating the primality of numbers of the form $2n^2 - 1$}

We can use a sieve method to generate all primes.
This relies on two claims: \\

\textbf{Claim 1:} If $d$ is a non-trivial (i.e.~$d \neq 1$) divisor of $t(n) = 2n^2 - 1$, then
\begin{itemize}
\item[(a)] $d$ does not divide $n$, $n+1$, or $n-1$.
\item[(b)] $d$ divides $t(n + k t(n))$ for all $k \in \mathbb{Z}$.
\end{itemize}
\emph{Proof.} For (a), note that $t(n) = 2n^2 - 1 = 2(n \pm 1)^2 \mp 4(n \pm 1) + 1$ and $d$ cannot divide 1.
This guarantees that for no $k$ will $n + kt(n)$ be 0 or $\pm 1$.
For (b), note that $t(n + kt(n)) = t(n)\qty[ 1 + 2k(2n + kt(n))]$. $\blacksquare$ \\

\textbf{Claim 2:} Let $d$ be the smallest non-trivial divisor of $t(n) = 2n^2 - 1$.
If $d \neq t(n)$ then $d$ divides $t(n')$ for some $2 \leq n' < n$. \\

\emph{Proof.} We just need to prove that $d < 2n$. 
Then, from Claim 1, we have $2 \leq |n - d| < n$ and so $d$ divides $t(n-d)$. 
As $d$ is the smallest divisor of $t(n)$ but $d \neq t(n)$, we necessarily have $d^2 \leq t(n)$.
\[ d^2 \leq t(n) = 2n^2 - 1 < 2n^2 \qquad\implies\qquad d < \sqrt{2} n < 2 n \qquad \blacksquare\]

\subsection*{231 -- The prime factorisation of binomial coefficients} 
Kummer's theorem on binomial coefficients states that given integers $n \geq m \geq 0$ and a prime $p$ that the maximum integer $k$ such that $p^k$ divides $\binom{n}{m}$ is equal to the number of carries then $m$ is added to $n-m$ in base $p$.

\subsection*{235 -- An Arithmetic Geometric sequence} 
The sum of the first $n$ terms of a arithmetico-geometric sequence is given by
\[ \sum\limits_{k=1}^n [a+(k-1)d]r^{k-1} = \frac{a + (d-a)r - (a+nd)r^n +(a+nd-d)r^{n+1}}{(1-r)^2}\]

\subsection*{301$\star$ -- Nim} 
Given $k$ piles with $a_k$ stones, if $a_1 \wedge a_2 \wedge \dotsb \wedge a_k = 0$, then you have a losing game ($\wedge$ is the binary xor operation). 
For 3 piles, $a_1 \wedge a_2 \wedge a_3 = 0 \iff a_1 \wedge a_2 = a_3$. 
Then for this question, our requirement becomes $n \wedge 2n = 3n = n + 2n$. 
This is only true if the binary representation for $n$ does not have two adjacent ``1'' digits. 
For instance if $n = 1011_2$, $2n = 10110_2$ and $n \wedge 2n = 11101_2 \neq 100001_2 = n+2n$. 
So we are finding the number of binary strings of length $m$ with no adjacent ``1'' digits, which is given by the $m$-th Fibonacci number, with $F_1 = 2, F_2 = 3$. 
Thus using these seeds, $F_{30} = \boxed{2178309}$ gives our answer.

\subsection*{317$\star$ -- Firecracker}
This problem has a closed-form answer from physics.
Let us place the firecracker as it explodes at the origin, with the $y$-axis extending upwards and the $x$-axis extending in a horizontal direction.
Let us pick a height $H$, and we want to find the largest range $R$ for a projectile at this height.
So if a projectile is launched at angle $\theta$ to the horizon with speed $v$, we have the usual equations of motion,
\[  y(t) = v \cos\theta t - \frac{1}{2} gt^2 \qq{,} x(t) = v t \sin\theta \]
where $t$ is the time elapsed in seconds ($t=0$ at the origin) and $g$ is the acceleration due to gravity.
Combining these equations and eliminating $t$ gives,
\[ H (1 - \cos 2\theta) = R \sin 2 \theta - \frac{g R^2}{v^2} \]
where $R = R(\theta)$ is a function of $\theta$.
Now we differentiate both sides with respect to $\theta$ and set $\dv*{R}{\theta}=0$ to solve for the angle which maximizes the range, we get the condition $\tan 2\theta = R/H$.
Plugging this angle back into our previous equation, we get an expression for $R$,
\[ R = \sqrt{4 h_\text{max} (h_\text{max} - H)} \qq{,} h_\text{max} \equiv \frac{v^2}{2g} \]
where $h_\text{max}$ is the maximum height reached by the projectile.
For different values of $H$ between $h_\text{max}$ and $-h_0 \equiv - 100$, this gives the envelope for particles from the firecracker.
As the volume is rotationally symmetrical along its vertical axis, we can slice it up into small disks with radius $R$ and width $\dd{H}$.
So we can integrate to find the total volume,
\[ V = \int_{-h_0}^{h_\text{max}} \pi R^2 \dd{H} = \int_{-h_0}^{h_\text{max}} 4\pi {h_\text{max}}({h_\text{max}} - H) \dd{H} = 2 \pi h_\text{max} (h_\text{max} + h_0)^2 \]
If we plug-in the values ${h_\text{max}} = 20^2/2 \times 9.81$ and $h_0 = 100$, we have the answer,
\[ \boxed{ 2 \pi \qty(\frac{20^2}{2 \cdot 9.81}) \qty(\frac{20^2}{2 \cdot 9.81} + 100)^2 } \]

\subsection*{381 -- (prime-k) factorial} 
Wilson's Theorem states that for prime $p$, $(p-1)! \equiv -1 $ mod $p$. 
Thus we can deduce that $(p-2)! \equiv 1$, $(p-3)! \equiv -1/2$, $(p-4)! \equiv 1/6$, and $(p-5)! \equiv -1/24$ mod $p$ (fractions are to be interpreted as modular inverses). 
Thus $S(p) \equiv -3/8 $ mod $p$.

\subsection*{389$\star$ -- Platonic Dice}
We are interested in finding a generating function $G(x)$ with the property,
\[ G(x) = \sum_{n=1} P_n x^n \]
where $n$ is the sum on the dice and $P_n$ is the probability of obtaining that sum.
For the normalization of probabilities, we require $G(x) = 1$.
We can see that for a single 4-sided die, the generating function is
\[ G(x) = \frac{1}{4}(x + x^2 + x^3 + x^4) \]
Now according to the problem statement, for each outcome of rolling a number in $\{1,2,3,4\}$ we roll that many 6-sided dice.
The generating function for this situation can be obtained by replacing
\[ x \to \frac{1}{6}(x+x^2+x^3+x^4+x^5+x^6) \]
or in other words, if we define,
\[ g_4(x) \equiv \frac{1}{4}(x+x^2+x^3+x^4) \qq{,} g_6(x) \equiv \frac{1}{6}(x+x^2+x^3+x^4+x^5+x^6)\]
then the new generating function is
\[ G(x) = g_4 (g_6(x)) = g_4 \circ g_6(x) \]
This works because the powers of $x$ in the generating function tell us both the sum on the dice and the number of new rolls to perform.
In general, if we define,
\[ g_n(x) \equiv \frac{1}{n}(x+x^2 + x^3 + \cdots + x^n) \]
then the generating function for this problem is
\[ G(x) = g_4 \circ g_6 \circ g_8 \circ g_{12} \circ g_{20} (x) \]
We want to calculate the variance of $I$, given by $\sigma^2 = \ev{I^2} - \ev{I}^2$.
This can be calculated from our generating functions,
\[ \ev{I} = \sum_{n=1} n P_n = \dv{x} \qty( \sum_{n=1} P_n x^n)_{x=1} = G'(1) \]
\[ \ev{I^2} = \sum_{n=1} n^2 P_n = \dv{x} \qty( x \dv{x} G(x) )_{x=1} = G'(1) + G''(1) \]
Using the normal derivative rules,
\[ G'(x) = g'_{20}(x) \times g'_{12} \circ g_{20}(x) \times g'_8 \circ g_{12} \circ g_{20}(x) \times g'_6 \circ g_8 \circ g_{12} \circ g_{20}(x) \times g'_4 \circ g_6 \circ g_8 \circ g_{12} \circ g_{20}(x)\]
Note that $g_n(1) = 1$, so
\[ G'(1) = g'_4(1) \times g'_6(1) \times g'_8(1) \times g'_{12}(1) \times g'_{20}(1) \]
Likewise, you can show that,
\[  G''(1) = G'(1)\qty( \frac{g''_{20}}{g'_{20}} + g'_{20} \qty(\frac{g''_{12}}{g'_{12}} + g'_{12} \qty(\frac{g''_8}{g'_8} + g'_8 \qty(\frac{g''_6}{g'_6} + g'_6 \cdot \frac{g''_4}{g'_4}))))_{x=1}  \]
Now you can also work out that,
\[ g'_n(1) = \frac{1}{2}(n+1) \qq{,} \frac{g''_n(1)}{g'_n(1)} = \frac{2}{3}(n-1) \]
Plugging everything in, this gives,
\[ G'(1) = \frac{85995}{32} \qq{,} G''(1) = G'(1) \cdot \frac{85963}{24} \]
And the final answer for $\sigma^2$ is,
\[ \frac{2464129395}{1024} \approx \boxed{ 2406376.3623} \]

\subsection*{407 -- Idempotents} 
For a particular $n$, let is prime factorization be $n = p_1^{a_1} p_2^{a_2} \cdots p_k^{a_k}$ for distinct primes $p_1, \dotsc , p_k$.
The Chinese Remainder Theorem defines a ring isomorphism,
\[ \mathbf{Z}_n \cong \mathbf{Z}_{p_1^{a_1}} \oplus  \mathbf{Z}_{p_2^{a_2}} \oplus \cdots \oplus \mathbf{Z}_{p_k^{a_k}} \]
The only idempotents in $\mathbf{Z}_{p^a}$ for any prime $p$ are 0 and 1.
Therefore, $\mathbf{Z}_n$ has $2^k$ idempotents corresponding to every possible selection of 0 or 1 in each $\mathbf{Z}_{p_i^{a_i}}$, i.e.
\[ \underbrace{(0, 0, \dotsc, 0)}_{k \text{-tuple}} \quad,\quad (1, 0, \dotsc, 0) \quad,\quad (0, 1, \dotsc, 0) \quad , \quad \dotsc \quad , \quad (1, 1, \dotsc, 1) \]
We use the isomorphism defined by the Chinese Remainder Theorem to convert elements of $\bigoplus_i \mathbf{Z}_{p_i^{a_i}}$ into elements of $\mathbf{Z}_n $.


\subsection*{500 -- Problem 500!!} 
Recall that given the prime factorization of a number $n = p_1^{a_1} p_2^{a_2} \dotsb p_k^{a_k}$, the number of divisors of $n$ is $(a_1 + 1)(a_2 + 1)\cdots(a_k+1)$. 
So a number with $2^{500500}$ divisors is $2\times 3\times\cdots\times p_{500500}$. 
However, this is not the smallest number, as we can replace $p_{500500}$ with $2^2$ to get $2^3\times3\times\cdots\times p_{500499}$. 
We repeat this, replacing the largest primes in the product with powers of smaller primes.

\subsection*{577$\star$ -- Counting hexagons}
With length $n=3$, we are able to construct a hexagon from the center point with ``radius 1'' (i.e.~the vertices are a unit distance away from the center).
If the side length is increased to $n=4$, then we can construct 3 such hexagons, then 6 such hexagons with $n=5$, and so forth.
This pattern of triangular numbers continues: with side length $n$ we are able to construct $T_{n-2}$ hexagons with radius 1, where $T_n = n(n+1)/2$ is the $n$th triangular number. \\

However, at $n=6$, we are able to construct two new types of hexagons from the center point with ``radius 2'' (i.e.~the vertices are two units away from the center, counting distances along the edges of the equilateral triangles). 
We can construct one of each type, so including the 10 hexagons with radius 1, we have 12 total hexagons, hence $H(6)=12$.
With $n=7$, we are able to construct 3 of each type of hexagon with radius 2, and adding the 15 hexagons with radius 1, we have $15 + 2\times 3 = 21$ total hexagons, hence $H(7)=21$.
Again, these radius 2 hexagons follow the same triangular number pattern: with side length $n$ we are able to construct $2T_{n-5}$ hexagons with radius 2. \\

Then at $n=9$, we are able to construct three new types of hexagons from the center point with ``radius 3''.
This pattern also continues: when the side length $n$ is a multiple of 3, we are able to construct $n/3$ new types of hexagons.
These new hexagons follow the same triangular number pattern. \\

To give a sample calculation, in calculating $H(20)$, we can have up to radius 6 hexagons.
The total number of hexagons we can construct is,
\[ T_{18} + 2T_{15} + 3 T_{12} + 4 T_{9} + 5 T_{6} + 6 T_{3} = 966 \]
Using this, we can express our sum as a sum over triangular numbers:
\[ \sum_{n=3}^{12345} H(n) = \sum_{n=1}^{12345-2} T_n + 2\times \sum_{n=1}^{12345-5}T_n + 3 \times \sum_{n=1}^{12345-8} T_n + \cdots + 4115 \times \sum_{n=1}^1 T_n \]
We will use the following formulas,
\begin{align*}
\sum_{n=1}^N n &= \frac{N(N+1)}{2} & \sum_{n=1}^N n^2 &= \frac{N(N+1)(2N+1)}{6} \\
\sum_{n=1}^N n^3 &= \frac{N^2 (N+1)^2}{4} & \sum_{n=1}^N n^4 &= \frac{N(N+1)(2N+1)(3N^2 + 3N - 1)}{30}
\end{align*}
So we can write,
\[ \sum_{n=1}^N T_n = \frac{N(N+1)(N+2)}{6} \]
This implies,
\begin{align*}
\sum_{n=3}^{12345} H(n) &= 4115 \frac{1(1+1)(1+2)}{6} + 4114 \frac{4(4+1)(4+2)}{6} + \cdots +\frac{12343(12343+1)(12343+2)}{6} \\
&= \frac{1}{6} \sum_{n=0}^{4114} (4115-n)(1+3n)(2+3n)(3+3n)
\end{align*}
Expanding this out and using the sums for $n, n^2, n^3, n^4$ above, we get the answer $\boxed{ 265695031399260211} $.

\subsection*{587 -- Concave triangle}
If the circle has unit radius, then the L-section has area $1 - \pi/4$.
If the corner of the L-section is at the origin, then the straight line is given by the equation $y = nx$ for a particular value of $n$.
The intersect of this straight line and the circle occurs at the $x$-value,
\[ x_n = \frac{n+1-\sqrt{2n}}{n^2+1} \]
Then using calculus, the area of the concave triangle is,
\[ \int_0^{x_n} \dd{x} \qty[(1 - \sqrt{1 - (x-1)^2}) - nx]  = \frac{1}{2} \qty[ (1-x_n) \sqrt{1 - (1 - x_n)^2} + \sin^{-1}(1-x_n) - nx_n^2 + 2x_n - \frac{\pi}{2} ] \]
We iterate through values of $n$ until the area is small enough.

\end{document}
